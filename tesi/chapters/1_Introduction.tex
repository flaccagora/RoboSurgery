
\chapter{Introduction}

\gls{ai} applications in robotic surgery represent a transformative frontier in the medical field, 
enhancing precision, safety, and efficiency in surgical procedures. 
The integration of such technologies with robotic systems has significantly advanced minimally 
invasive surgery, allowing for smaller incisions, reduced blood loss, and faster recovery times, 
while enabling complex procedures that require exceptional dexterity and visualization capabilities.

Notable robotic systems, such as the da Vinci Surgical System, exemplify the evolution of surgical 
practice through this innovative marriage of AI and robotics, paving the way for improved 
patient outcomes across various specialties, including urology, gynecology, and cardiothoracic surgery.

AI enhances several aspects of surgical practice, including preoperative planning, 
real-time decision support, and postoperative analysis, thereby transforming the surgical landscape.

These advancements underscore AI's potential to refine surgical techniques and foster a culture 
of continuous improvement in surgical care. As the field progresses, the future of AI in robotic 
surgery promises further innovations, such as remote surgical capabilities and enhanced imaging 
technologies, which could redefine healthcare access and delivery on a global scale.



The intersection of AI and robotic surgery not only marks a significant evolution in surgical practices but also holds the potential to fundamentally reshape patient care and outcomes in the coming years.
\gls{rl} in thoracic microinvasive surgery represents a pioneering intersection of artificial intelligence 
and surgical practice, aiming to enhance surgical precision and decision-making processes. 
As robotic-assisted surgical techniques gain traction, particularly through methods like 
\gls{vats} and \gls{rats}, 
the integration of RL algorithms into these workflows has become notable for its potential to 
transform patient outcomes. This application is increasingly relevant as the demand for minimally 
invasive approaches grows, driven by their associated benefits such as reduced postoperative pain 
and shorter recovery times.

The development of \gls{rl} techniques in this field leverages the ability of these algorithms to learn 
from complex interactions within dynamic surgical environments. By training robotic systems to 
optimize intricate tasks through feedback mechanisms, RL not only enhances the operational efficiency 
of surgical robots but also contributes to more informed preoperative planning through predictive 
analytics. 
These advancements allow for tailored risk assessments that improve decision-making in patient care.


Despite its promise, integration of such techniques into thoracic microinvasive surgery faces
challenges, including the need for robust regulatory frameworks and ongoing ethical scrutiny. 
Discussions surrounding the implications of AI in surgical contexts emphasize the importance of 
ensuring that healthcare providers remain central in the decision-making process while leveraging AI 
technologies to enhance patient care.

Overall, the ongoing research and development in reinforcement learning for thoracic microinvasive 
surgery exemplify a transformative potential in the healthcare landscape, with opportunities for 
improving surgical techniques, outcomes, and training methods while navigating the complexities 
of integrating AI into human-centered care.

% Methodology
\section{Problem Statement}
Over the last decade, 3D models have entered oncologic surgery
as a means to achieve better outcomes in renal and hepatic
surgery [1, 2]. Nevertheless, the integration of 3D models into
the operative field has been lacking due to three main reasons.
Firstly, proper model alignment with the intraoperative anatomy
has proven to be a major challenge due to shifting of organs
during surgery and different patient positioning in surgery ver-
sus during computed tomography [2, 3]. Secondly, automated
organ registra-tion in a continuously moving surgical video
has been another major challenge for many years [4]. Thirdly,
3D model overlay obscures the surgical field, including sharp
surgical instruments which are manipulated, hence creating a
possible hazardous situation rather than facilitating surgery. The
latter occlusion problem has been a longstanding study topic
[5] which, if solved, would further advance various surgical
domains and applications [6]. Already in 2004, Fischer et al. [7]

\paragraph{Navigation in deformable environments}
% Related Work
\section{Related Work}
% Deformable registration refers to the process of aligning multiple three-dimensional 
% images into a common coordinate frame. It quantifies changes in organ shape, size,
% and position, providing a comprehensive understanding of patient anatomy and 
% function. This technique is particularly useful in image-guided surgery and 
% radiotherapy to improve the accuracy of treatments.
\begin{itemize}
    \item Instrument occlusion
    \item DL recognition of anatomical structures
    \item Elastic Regristration
    \item RL for simple surgical tasks
\end{itemize}

While robotic systems have revolutionized minimally invasive surgery, they lack the 
ability to adapt to unexpected intraoperative changes autonomously. \gls{pomdp}s, with 
their capacity to model uncertainty, and RL, with its learning-based adaptability, 
present a promising synergy for addressing these challenges. However, their 
application to real-time intraoperative decision-making remains unexplored.


\paragraph{RL in Surgical Robotics}
\textbf{preso da} \citet{10578312}
ALarge number of recent research in surgical robotics
focuses on the automation of surgical sub-tasks, such
as needle manipulation [1], [2], [3], suturing [4], [5], cutting
[6], [7], [8], vessel manipulation [9], tissue retraction and
deformation [10], [11], [12], [13]. However, most of these
studies focus on the manipulation of rigid and soft objects,
although fluid-related tasks are also common in surgeries, due
to the presence of body fluids, especially blood 


\part{Theoretical Background}

