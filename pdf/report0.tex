\documentclass[12pt,a4paper]{article}

% Packages
\usepackage[utf8]{inputenc}    % Unicode support
\usepackage{amsmath,amsfonts}  % Math symbols
\usepackage{graphicx}          % Images
\usepackage{hyperref}          % Hyperlinks
\usepackage{geometry}          % Page layout
\usepackage{caption}           % Captions for figures/tables
\usepackage{cite}              % Citations


% Page layout settings
\geometry{
    top=1in,
    bottom=1in,
    left=1in,
    right=1in
}

% Title information
\title{RoboSurgery}
\author{Matteo Nunziante\thanks{matteo.nunziante@studenti.units.it}}
\date{\today}

\begin{document}

% Title
\maketitle

% Abstract
\begin{abstract}

\end{abstract}

% Keywords
\textbf{Keywords:} keyword1, keyword2, keyword3
\newpage


% Table of contents
\tableofcontents 
\newpage

% Introduction
\section{Introduction}
The goal of this project is to simulate a surgical robot for exploring lungs of patients.
% Related Work
\section{Related Work}

% Methodology
\section{Probkem Statement}


% POMDP
\section{POMDP}
Formal description of Partially Observable Markov Decision Process as in \cite{Spaan2012}

Formally, a POMDP is a 7-tuple $(S,A,T,R,\Omega,O,\gamma)$, where: \\
  $S$ is a set of states, \\
  $A$ is a set of actions, \\
  $T$ is a set of conditional transition probabilities between states, \\
  $R: S\times A \rightarrow \mathbb{R}$  is the reward function. \\
  $\Omega$ is a set of observations, \\
  $O$ is a set of conditional observation probabilities, \\
  $\gamma \in [0,1)$ is the discount factor.

\newpage
\section{Mathematical Formulation}
\subsection{State Space}
From now on we're going to indicate as \textbf{State} the fully observable state of the 
system, which is the position of the robot, the deformation of the object and the

$$s = (\underbar{pos},\theta)$$

where $\underbar{pos} = (x,y,\phi)$ is the position of the robot in the 2D space
and $\theta$ represents the deformation parameters of the object.

\subsection{Action Space}
The action space is the set of all possible actions that the robot can take, 
Forwards, Backwards, Left, Right.
\subsection{Observation Space}
The observation space is the set of all possible observations that the robot can make,
which is the presence of an obstacle in its field of view.
% Conclusion
\newpage
\section{Results}
\subsection{QMDP}
\subsection{Thompson Sampling}
\subsection{Infotaxis}
\subsection{Information Directed Sampling}
\cite{russo2017learningoptimizeinformationdirectedsampling}




% References
\newpage
\bibliographystyle{plain}
\bibliography{ref}


\end{document}
